\documentclass{article}
\usepackage{helvet}

\usepackage{listings}
\usepackage{fullpage}
\usepackage{color}
\usepackage{hyperref}
\usepackage{graphics}
\usepackage{graphicx}

\usepackage[utf8]{inputenc}
\usepackage[T1]{fontenc}

\newcommand{\question}{{\noindent \bf Question: }}
\newcommand{\todo}[1]{{\bf #1}}

\title{\vspace{-3em}Assignment: basic programming\\Numerical integration}

\date{}

\begin{document}

\maketitle

The purpose of this assignment is for you to 
\begin{itemize}
\item get started back in writing C or C++ code
\item get familiar with running code on mamba
\item write a simple program that will be reused in future assignments
\end{itemize}

As usual all time measurements are to be performed on the cluster.

\section{Preliminary: running anything on mamba}

\question Write a code that simply prints the name of the current
machine. You can obtain that name using function \texttt{gethostname}
(check man page for details). Write that code in file
\texttt{prelim.cpp} or \texttt{prelim.c}. You can test this works by compiling it with \texttt{make prelim}.

\question Run that code on a mamba compute node using
\texttt{./queue\_prelim.sh}. This will start a PBS job. Once this job
has completed, a file \texttt{preliminary\_answer} will be created
containing the name of the compute node the command has run on.

\section{Numerical Integration}

Numerical integration is often used when one wants to compute
$\int_{a}^{b} f(x) dx$ but one does not know how to find a primitive
of $f$. You can use the definition of integration to obtain a simple
approximation by computing $\frac{b-a}{n} \sum_{i=0}^{n-1}
f\left(a+(i+.5)*\frac{b-a}{n}\right)$. $n$ is often called
the number of point in the approximation. (This is the numerical
integration using the rectangle rule. You can learn more at
\url{https://en.wikipedia.org/wiki/Numerical_integration}.)

{\bf Note that you do not need to understand numerical
integration.} The problem is just to evaluate $\frac{b-a}{n} \sum_{i=0}^{n-1}
f \left( a+(i+.5)*\frac{b-a}{n} \right)$ for a particular
combination of $a$, $b$, $n$, and $f$.

The provided package contains multiple functions to integrate in
\texttt{libfunctions.a}. The functions are named \texttt{f1},
\texttt{f2}, \texttt{f3}, \texttt{f4}, and take two parameters: the
first one is a floating point number $x$ where the function is
computed, and the second one is $intensity$ an operation
intensity. The second parameter is used to make the function take more
time.

The code you should write should take 5 command line parameters:
\begin{itemize}
\item \texttt{functionid}, an integer to know which function to integrate. If \texttt{functionid} is 1, integrate \texttt{f1}
\item \texttt{a}, the lower bound of the integral
\item \texttt{b}, the upper bound of the integral
\item \texttt{n}, an integer which is the number of points to compute the approximation of the integral
\item \texttt{intensity}, an integer which is the second parameter to give the the function to integrate
\end{itemize}

The code should compute the integral and output the value of the
integral on \texttt{stdout} (and nothing else). The code should also
measure the time it took to compute the integral and write that time
(in seconds) to \texttt{stderr}.

\question Write the described code. You can use the provided archive
as a template. It contain a template code and makefile to help you
write the code. You should only need to complete
\texttt{main.cpp}. You should be able to test if your code is correct
using \texttt{make test}.

\question Report the time it takes on the cluster to integrate
\texttt{f1} using different number of points (from $10^1$ to $10^8$)
and with different operation intensity (from $1$ to $10^4$). To help
you in that task, you should be able to run \texttt{make bench} which
should run the benchmark in a PBS job. Once that job is completed, you
can draw charts using \texttt{make plot} which reports time in a pdf
file \texttt{plot/time\_plots.pdf}.

Make sure you keep this code around as it is your base for comparisons
in future assignments. Also note that a run with $10^8$ points and an
operation intensity of $1,000$ could take an hour to run.

\end{document}
